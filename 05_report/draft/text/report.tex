\documentclass[xelatex,ja=standard]{bxjsarticle}
\input{01_admin/preamble/tex/admin.tex}
\usepackage{fontspec}
\usepackage{underscore}
\usepackage[T1]{fontenc}
  
\setCJKsansfont[Path = ../../../\detokenize{01_admin}/\detokenize{02_preamble}/tex/fonts/,
BoldFont = HaranoAjiGothic-Bold.otf]
{HaranoAjiGothic-Regular.otf}

\setCJKmainfont[Path = ../../../\detokenize{01_admin}/\detokenize{02_preamble}/tex/fonts/,
BoldFont = HaranoAjiMincho-Bold.otf]
{HaranoAjiMincho-Regular.otf}

\usepackage{threeparttable}

\title{朝ごはんと成績に統計的関係があるか \\ -- Peanutsのデータを用いて -- \thanks{aaa}}

\author{Charlie Schultz
\thanks{Department of Comics. Email: michihito.ando@rikkyo.ac.jp}  \ \  古川知志雄
\textsuperscript\thanks{Department of Economics. Email:furukawa-chishio-gj@ynu.ac.jp}}

%\date{\today}
\date{March, 1990}


\begin{document}
\renewcommand\footnotelayout{\small}
\sffamily\mdseries

\maketitle

\vspace{-10pt}\begin{abstract}
\begin{spacing}{1}
\noindent 
Peanutsの登場キャラクターたちは、自分たちの経験に基づき「好きな朝ごはんを食べることが、学校の成績につながる」と考えています。本レポートでは、この統計的根拠を探索し、以下の結果を得ました。(1)朝ごはんに``pancake''を食べることは、小テストの成績と強く相関はしていなかった。(2)Snoopyが``corn flake''の代わりに``dog flake''ばかり買ったときには「朝ごはんに``dog flake''を食べ、アレルギーで勉強に集中できなかった」とCharlieとSallyは言っていますが、これについては整合的な根拠が示された。\\

\end{spacing}
\end{abstract}

\newpage

\section{相関関係の分析} 


\begin{figure}
\centering
\includegraphics[width=7cm]{04_analyze/scatter_regress/figure/figure.pdf}

\label{fig:img1}
\caption{Bare figure}
\end{figure}

\mcfamily\mdseries

\begin{table}

\caption{Initial regressions}
\centering
\begin{threeparttable}
\begin{tabular}[t]{lcc}
\toprule
\multicolumn{1}{c}{ } & \multicolumn{1}{c}{(1)} & \multicolumn{1}{c}{(2)} \\
\cmidrule(l{3pt}r{3pt}){2-2} \cmidrule(l{3pt}r{3pt}){3-3}
  & Average & Average \\
\midrule
freq(pancakes) & 2.42 & 2.19\\
 & (2.07) & (2.03)\\
Constant & 44.39 & \\
 & (3.15) & \\
\midrule
Num.Obs. & 111 & 111\\
R2 & 0.033 & 0.072\\
Model & OLS & FE\\
NA & NA & NA\\
Clustering & Y & Y\\
R2 Adj. & 0.024 & 0.046\\
\bottomrule
\end{tabular}
\begin{tablenotes}
\item \textit{Note: } 
\item Heteroskedasticity-robust standard errors clustered at students level are reported in the parenthesis.
\end{tablenotes}
\end{threeparttable}
\end{table}


\begin{table}[!h]
\centering
\begin{tabular}[t]{lrr}
\toprule
breakfast\_renamed & count & frequency\\
\midrule
cereal & 292 & 0.33\\
dog flakes & 6 & 0.01\\
no breakfast & 41 & 0.05\\
oatmeal & 206 & 0.23\\
pancakes & 178 & 0.20\\
\addlinespace
toast & 172 & 0.19\\
\bottomrule
\end{tabular}
\end{table}


Peanutsの登場キャラクターたちは、自分たちの経験に基づき「好きな朝ごはんを食べることが、学校の成績につながる」と考えています。本レポートでは、この統計的根拠を探索し、以下の結果を得ました。(1)朝ごはんに``pancake''を食べることは、小テストの成績と強く相関はしていなかった。(2)Snoopyが``corn flake''の代わりに``dog flake''ばかり買ったときには「朝ごはんに``dog flake''を食べ、アレルギーで勉強に集中できなかった」とCharlieとSallyは言っていますが、これについては整合的な根拠が示された。

% \begin{figure}
% \centering
% \includegraphics[width=7cm]{Unemployment_Insurance.pdf}

% \label{fig:img2}
% \caption{Simulation}
% \end{figure}

% \begin{table}
% \let\center\empty
% \let\endcenter\relax
% \centering
% \resizebox{.8\width}{!}{\input{subfolder/table}}
% \end{table}

\bibliographystyle{econ} 
\bibliography{01_admin/preamble/tex/literature.bib}

\end{document}